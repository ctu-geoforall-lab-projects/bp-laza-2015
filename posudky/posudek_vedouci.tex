\documentclass[czech,11pt,a4paper]{article}
\usepackage[utf8]{inputenc}
\usepackage{a4wide}
\usepackage[pdftex,breaklinks=true,colorlinks=true,urlcolor=blue,
  pagecolor=black,linkcolor=black]{hyperref}
\usepackage[czech]{babel}

\pagestyle{empty}

\begin{document}

\begin{center}
  {\Large --- Posudek vedoucího bakalářské práce ---}
\end{center}

\vspace{.2cm}

\noindent \begin{tabular}{rp{.75\textwidth}}
  {\bf Bakalářská práce:} & Implementace nástroje pro interpolaci metodou přirozeného souseda do GRASS GIS \\
  {\bf Student:} & Adam Laža \\
  {\bf Vedoucí:} & Ing. Martin Landa, Ph.D. \\
  {\bf Oponent:} & Ing. Tomáš Bayer, Ph.D. \\
\end{tabular}

\vspace{1cm}

Zadání bakalářské práce Adama Laži vychází z praktického potřeby
rozšířit open source projekt GRASS GIS o nativní implementaci nástroje
pro interpolaci metodou přirozeného souseda. GRASS GIS podobným
nástrojem již disponuje, ale pouze v tzv. formě AddOns s~externí
závislostí na knihovnu Triangle, která díky své licenční politice není
běžně rozšířena v Linuxových distribucích a nelze ji šířit snadno ani
jako součást instalace pod OS MS Windows. To činí tento nástroj pro
běžné uživatele systému GRASS prakticky nedostupným.
\\

Student musel v úvodu své práce nastudovat teorii z oblasti výpočetní
geometrie a seznámit se s prostředím nástroje GRASS GIS především z
pohledu programátora a vývojáře. V~první kroku přepsal již existující
nástroje dostupné pro GRASS jako POSIX skripty do jazyka Python. To
umožnilo začlenění těchto nástrojů do nové verze GRASS 7, která již
POSIX skripty nepodporuje. Závislost na knihovně Tringle, resp. nn-c
zůstala zachována. Student se musel naučit základy programování v
jazyce Python včetně objektového návrhu s využitím knihoven
GRASS. Výsledek byl úspěšně začleněn do GRASS jako AddOns.
\\

V dalším kroku se student podrobně věnoval možnosti nahrazení
problematické závislosti na knihovně Triangle kódem, který je v
dostupný v knihovnách systému GRASS. Tato myšlenka se ukázala jako
realizovatelná, nicméně vzhledem k pracnosti a očekávatelnému výsledku
bylo od této cesty upuštěno. Student v této oblasti načerpal znalosti
v oblasti analýzy kódu a teoretickém základu, který byl třeba k
pochopení této problematiky.
\\

Na závěr se student věnoval implementaci nástroje s využitím další
knihovny CGAL. Tato knihovna je napsána v~programovacím jazyce C++ a
je šířena pod licencí, která je kompatibilní se systémem GRASS. Zde
student projevil základní znalost programování v~C++. Výsledek je
prototyp funkčního nástroje, který ale trpí závažnými problémy v
rychlosti výpočtu a i jeho správnosti. Tato část práce by si zasloužila
více prostoru a může být vnímána jako prostor pro další pokračování
tohoto v tématu v rámci diplomové práce.
\\

Student projevil značnou schopnost samostatné práce (včetně diskuze na
mezinárodních fórech) a analytického pohledu. Z hlediska
programátorského pracoval student ve třech programovacích jazycích a
to POSIX (Bash), Python a C++. Výsledkem je práce, kterou lze označit
za nadprůměrnou a to i přes řadu nedostatků a to především v
teoretické části.
\\
\newpage
Závěrem mohu konstatovat, že předložená bakalářská práce splňuje
všechny formální náležitosti a doporučuji ji k obhajobě. Bakalářskou
práci hodnotím stupněm

\vskip 2cm

\begin{center}
{\bf -- B (velmi dobře)  --}
\end{center}

\vskip 2cm

\begin{tabular}{lp{.25\textwidth}r}
& & \ldots\ldots\ldots\ldots\ldots\ldots\ldots \\
V~Praze dne 10. června 2015 & & Ing. Martin Landa, Ph.D. \\
& & Fakulta stavební, ČVUT v Praze \\
\end{tabular}

\end{document}
